\chapter{Verbesserungen}
Die hier beschriebene Umsetzung ist als Proof-Of-Concept zu verstehen. Es wurden keine Optimierungen vorgenommen und lediglich das Grundlegende Verfahren umgesetzt. Unter anderem wurde das G-Buffer Layout und seine Größe nicht mit den Anforderungen skaliert. Bis auf dem Depth-Attachment mit 24-Bit sind alle anderen Attachments 32-Bit Float-Texturen. Wie in Abbildung~\ref{fig:gbuffer} zu sehen ist, werden 64-Bit der Kapazität des G-Buffers nicht genutzt. Auch kann die Größe der einzelnen Attachments weiter reduziert werden. Beispielsweise ist zur Speicherung der Komponenten eines Normalen-Vektors keine 32-Bit Float-Textur nötig. Auch wird im Attachment das den Spekular-Koeffizienten enthält nur ein Kanal genutzt. Dieser könnte auch im Alpha-Kanal des Attachement welches die Albedo-Color enthält gespeichert werden.

Auch der Rendering-Prozess kann deutlich verbessert werden. Jede Punktlichtquelle beleuchtet nur einen Bereich innerhalb ihres Radius. Trotzdem wird für jedes Fragment jede Lichtquelle in die Berechnung einbezogen, auch wenn diese keinen Beitrag zur Beleuchtung des Fragments gibt. Dies lässt sich Beispielsweise dadurch verhinden, das für jedes Licht ein Quadrat mit Kantenlänge seines Durchmessers im Screen-Space gerendert wird und nur auf diesem Quadrat mit dieser Lichtquelle die Beleuchtung berechnet. Die einzelnen Quadrate werden dann additiv geblendet um das korrekte Bild zu erhalten.

Außerdem werden bisher auch Lichtquellen in die Berechnung einbezogen die außerhalb des View-Frustums der Kamera liegen. Durch eine Art Frustum-Culling lässt sich auch hier Rechenleistung einsparen.