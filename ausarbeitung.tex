%%%%%%%%%%%%%%%%%%% vorlage.tex %%%%%%%%%%%%%%%%%%%%%%%%%%%%%
%
% LaTeX-Vorlage zur Erstellung von Projekt-Dokumentationen
% im Fachbereich Informatik der Hochschule Trier
%
% Basis: Vorlage svmono des Springer Verlags
%
%%%%%%%%%%%%%%%%%%%%%%%%%%%%%%%%%%%%%%%%%%%%%%%%%%%%%%%%%%%%%

\documentclass[envcountsame,envcountchap, deutsch]{i-studis}

\usepackage{makeidx}         	% Index
\usepackage{multicol}        	% Zweispaltiger Index
%\usepackage[bottom]{footmisc}	% Erzeugung von Fußnoten

%%-----------------------------------------------------
%\newif\ifpdf
%\ifx\pdfoutput\undefined
%\pdffalse
%\else
%\pdfoutput=1
%\pdftrue
%\fi
%%--------------------------------------------------------
%\ifpdf
\usepackage[pdftex]{graphicx}
\usepackage[pdftex,plainpages=false]{hyperref}
%\else
%\usepackage{graphicx}
%\usepackage[plainpages=false]{hyperref}
%\fi

%%-----------------------------------------------------
\usepackage{color}				% Farbverwaltung
%\usepackage{ngerman} 			% Neue deutsche Rechtsschreibung
\usepackage[english, ngerman]{babel}
%\usepackage[latin1]{inputenc} 	% Ermöglicht Umlaute-Darstellung
\usepackage[utf8]{inputenc}  	% Ermöglicht Umlaute-Darstellung unter Linux (je nach verwendetem Format)

%-----------------------------------------------------
\usepackage{listings} 			% Code-Darstellung
\lstset
{
	basicstyle=\scriptsize, 	% print whole listing small
	keywordstyle=\color{blue}\bfseries,
								% underlined bold black keywords
	identifierstyle=, 			% nothing happens
	commentstyle=\color{red}, 	% white comments
	stringstyle=\ttfamily, 		% typewriter type for strings
	showstringspaces=false, 	% no special string spaces
	framexleftmargin=7mm, 
	tabsize=3,
	showtabs=false,
	frame=single, 
	rulesepcolor=\color{blue},
	numbers=left,
	linewidth=146mm,
	xleftmargin=8mm
}
\usepackage{textcomp} 			% Celsius-Darstellung
\usepackage{amssymb,amsfonts,amstext,amsmath}	% Mathematische Symbole
\usepackage[german, ruled, vlined]{algorithm2e}
\usepackage[a4paper]{geometry} % Andere Formatierung
\usepackage{bibgerm}
\usepackage{array}
\usepackage{mdframed}
\let\proof\relax
\let\endproof\relax
\usepackage{amsthm}
\usepackage{tikz}
\usetikzlibrary{matrix}
\usetikzlibrary{calc}
\usepackage{tikz-uml}
\usepackage{caption}
\usepackage{subcaption}
\usepackage{enumitem}
\hyphenation{Ele-men-tar-ob-jek-te  ab-ge-tas-tet Aus-wer-tung House-holder-Matrix Le-ast-Squa-res-Al-go-ri-th-men} 		% Weitere Silbentrennung bei Bedarf angeben
\setlength{\textheight}{1.1\textheight}
\pagestyle{myheadings} 			% Erzeugt selbstdefinierte Kopfzeile
\makeindex 						% Index-Erstellung

\lstdefinestyle{cpp}{
language=C++,
basicstyle=\ttfamily\small\color{black},
keywordstyle=\color{blue},
stringstyle=\color{yellow!20!black},
showstringspaces=false,
identifierstyle=\color{black},
commentstyle=\color{green!40!black},
tabsize=4,
escapechar=$
}

%--------------------------------------------------------------------------
\begin{document}
%------------------------- Titelblatt -------------------------------------
\title{Deferred Shading}
%\subtitle{English Title}
%---- Die Art der Dokumentation kann hier ausgewählt werden---------------
%\project{Bachelor-Projektarbeit}
%\project{Bachelor-Abschlussarbeit}
%\project{Master-Projektstudium}
%\project{Master-Abschlussarbeit}
%\project{Bachelor-Seminar}
\project{Hausarbeit zur Vorlesung Realtime-Rendering}%--------------------------------------------------------------------------
\supervisor{Prof. Dr. Rezk-Salama} 		% Betreuer der Arbeit
\author{Georg Schäfer} 							% Autor der Arbeit
\address{Trier,} 							% Im Zusammenhang mit dem Datum wird hinter dem Ort ein Komma angegeben
\submitdate{31.03.2015} 				% Abgabedatum
%\begingroup
%  \renewcommand{\thepage}{title}
%  \mytitlepage
%  \newpage
%\endgroup
\begingroup
  \renewcommand{\thepage}{Titel}
  \mytitlepage
  \newpage
\endgroup
%--------------------------------------------------------------------------
\frontmatter 
%--------------------------------------------------------------------------
%\chapter*{Kurzfassung}
\tableofcontents 						% Inhaltsverzeichnis
%--------------------------------------------------------------------------
\mainmatter                        		% Hauptteil (ab hier arab. Seitenzahlen)
%--------------------------------------------------------------------------
% Die Kapitel werden in separaten .tex-Dateien abgelegt und hier eingebunden.
% ...
\chapter{Funktionsweise}
Deferred-Shading ist eine Rendering Technik die den Shading-Vorgang, wie etwa die Beleuchtungsberechnung, im Screen-Space durchführt. Im Gegensatz zum klassischen Forward-Shading sind die Berechnungen unabhängig von der Szenenkomplexität und ermöglichen somit unter anderem aufendiger beleuchtete Bilder.

Beim Forward-Shading wird die Geometrie in den Vertex-Shader gegeben, Transformationen und ähnliches angewandt und anschließend an den Rasterizer weiter gereicht, der die Fragmente berechnet die die Geometrie ausmachen. Im nächsten Schritt erhält der Fragment-Shader diese Fragmente und verarbeitet sie um die tatsächliche Farbe des Fragments zu bestimmen. Erst am Ende dieser Operationen wird mit dem Tiefentest festgestellt, ob das berechenete Fragment auf dem Bildschirm angezeigt wird \ref{fig:forward}. Das hat zur Folge, dass das Shading einer Szene von der Komplexität der Geometrie in der Szene abhängig ist. Demnach ist eine Szenenbeleuchtung, mit einer großen Anzahl an Lichtquellen, mittels Forward-Shading sehr rechenaufwändig.


\begin{figure}
\begin{tikzpicture}
\draw[->, very thick] (0,0.5) -- (2, 0.5) node[pos=.5, above] {Geometry};
\draw (2,0) rectangle (5,1) node[pos=.5]{Vertex Shader};
\draw[->, very thick] (5,0.5) -- (8, 0.5) node[pos=.5, above] {Rasterization};
\draw (8,0) rectangle (12,1) node[pos=.5]{Fragment Shader};
\draw[->, very thick] (12,0.5) -- (14, 0.5) node[pos=.5, above] {Image};
\end{tikzpicture}
\caption{Forward-Rendering Pass}
\label{fig:forward}
\end{figure}

\footnote{Zur besseren Veranschaulichung wurde in obigem Diagramm auf die Operationen die nach Ausführung des Fragment-Shaders, sowie der Shading-Stages die zwischen Vertex- und Fragment-Shader liegen können, verzichtet}

Deferred-Shading ist ein Ansatz um dieses Problem zu lösen. Dabei wird der Render-Vorgang in zwei Schritte geteilt. Im ersten Schritt wird die Geometrie mit allen für die Beleuchtung benötigten Informationen, wie etwa der Frabe, der Normalen und Material-Eigenschaften, in einen Buffer gerendert \textendash{} den sog. Geometry-Buffer. Im zweiten Schritt wird nun die Beleuchtung, mithilfe des G-Buffers, im Screen-Space berechnet. Dazu wird ein Quadrat auf größe des Bildschirms gerendert, das das fertige Bild enthält \ref{fig:deferred}. An der Beleuchtungsberechnung selbst ändert sich beim Deferred-Shading nichts. Im zweiten Pass ist es nicht mehr möglich direkt an die Position der Vertices zu gelangen, allerdings lässt sich mithilfe des Depth-Buffers und den Fragmentkoordinaten im Screen-Space die frühere Position des Fragments innerhalb der Szene ermitteln. So wird Bandbreite und Speicher auf der Grafikkarte gespart, da die Fragementpositionen nicht mit im G-Buffer gespeichert werden müssen.


\begin{figure}
\begin{subfigure}[b]{0.5\textheight}
\begin{tikzpicture}
\draw[->, very thick] (0,0.5) -- (2, 0.5) node[pos=.5, above] {Geometry};
\draw (2,0) rectangle (5,1) node[pos=.5]{Vertex Shader};
\draw[->, very thick] (5,0.5) -- (8, 0.5) node[pos=.5, above] {Rasterization};
\draw (8,0) rectangle (12,1) node[pos=.5]{Fragment Shader};
\draw[->, very thick] (12,0.5) -- (14, 0.5) node[pos=.5, above] {G-Buffer};
\end{tikzpicture}
\caption{First Pass}
\end{subfigure}

\begin{subfigure}[b]{0.5\textheight}
\begin{tikzpicture}
\draw[->, very thick] (0,4.5) -- (5, 4.5) node[pos=.5, above] {G-Buffer, Lights};
\draw (5,4) rectangle (9,5) node[pos=.5]{Fragment Shader};
\draw[->, very thick] (9,4.5) -- (14, 4.5) node[pos=.5, above] {Image};
\end{tikzpicture}
\caption{Second Pass}
\end{subfigure}
\caption{Deferred-Rendering Pass}
\label{fig:deferred}
\end{figure}
\footnote{Zur besseren Veranschaulichung wurden in der obigen Darstellung auf einige Schritte verzichtet}
\chapter{Aufbau}
Zur Umsetzung des Deferred-Shadings wurde OpenGL als Grafikschnittstelle verwendet. Die nötige Kontext-Erstellung und das laden der OpenGL-Extensions wurden SDL und GLEW genutzt. Nach der Erstellung eines Fensters und der Initialisierung von OpenGL, werden zunächst alle benötigten Resource, wie etwa die verwendeten Shader, Texturen und Modelle geladen. Außerdem wird der G-Buffer mithilfe eines Framebuffer angelegt. Diesem werden vier Texturen, den sog. Framebuffer-Attachments, zur Speicherung der verschiedenen Informationen die für den Shading-Vorgang benötigt werden, hinzugefügt. In das erste Attachment wird der Depth-Buffer gerendert. Die übrigen drei Attachments beinhalten die Albdeo-Color, Normalen in Weltkoordinaten und den Spekular-Koeffizienten.

Das Layout des G-Buffers ist in \ref{fig:gbuffer} abgebildet.

\begin{figure}
\center
\begin{tikzpicture}
\draw (0,0) rectangle (3,-1) node[pos=.5]{32-bit};
\draw (3,0) rectangle (6,-1) node[pos=.5]{32-bit};
\draw (6,0) rectangle (9,-1) node[pos=.5]{32-bit};
\draw (9,0) rectangle (12,-1) node[pos=.5]{32-bit};
\draw (0,-1) rectangle (2.25,-2) node[pos=.5]{depth};
\draw (0,-2) rectangle (9,-3) node[pos=.5]{albedo color};
\draw (9,-2) rectangle (12,-3) node[pos=.5]{unused};
\draw (0,-3) rectangle (9,-4) node[pos=.5]{normal world};
\draw (9,-3) rectangle (12,-4) node[pos=.5]{unused};
\draw (0,-4) rectangle (3,-5) node[pos=.5]{specular};
\draw (3,-4) rectangle (12,-5) node[pos=.5]{unused};
\end{tikzpicture}
\caption{G-Buffer Layout}
\label{fig:gbuffer}
\end{figure}

In der Renderfunktion wird die Szenen-Geometrie nun gerendert und im Fragment-Shader die einzelnen Komponenten des G-Buffers befüllt. Lediglich der Depth-Buffer muss nicht selbst befüllt werden, da Framebuffer in OpenGL nur ein Depth-Attachment haben können und dieses als Depth-Buffer verwendet wird, sobald der Framebuffer gebunden wurde. Nach dem füllen des G-Buffers wird der Kompositionsschritt durchgeführt, mit dem das finale Bild berechnet wird. Hierzu wird ein Rechteck mit der Größe des Fensters gerendert und anschließend im Fragment-Shader die Beleuchtung berechnet. In diesen werden neben der Position, Farbe und dem Radius des Lichtes auch noch die einzelnen Attachments des G-Buffers und die Inverse View-Projection-Matrix übergeben. Diese wird benötigt um die frühere Position des Fragments in Weltkoordinaten zu ermitteln.

Dazu wird zunächst der Tiefenwert an der Stelle aus dem Depth-Attachment gelesen, die relativ zur Position des Fragments auf dem Bildschirm liegt. Dieser Wert liegt im Bereich [0,1], wobei 0 die Near-Plane ist und 1 die Far-Plane. Um die Z-Position in Weltkoordinaten zu erhalten muss der Tiefenwert in den Wertebereich [-1,1] gebracht werden.
Anschließend werden Fragmentposition und Tiefenwert in einem 4-Komponenten-Vektor zusammengefasst und die inverse View-Projection-Matrix mit diesem multipliziert. Der daraus resultierende Vektor wird durch seine vierte Komponente geteilt um die Position des Fragments in Weltkoordinaten zu erhalten. Nun kann die Berechnung der Beleuchtung erfolgen.

Für die Berechnung der Beleuchtung wurde das Modell von Blinn verwendet. Die verwendeten Lichtquellen sind ausschließlich Punkt-Lichtquellen. Außerdem wurde Tangent-Space-Normalmapping verwendet.
\chapter{Verbesserungen}
Die hier beschriebene Umsetzung ist als Proof-Of-Concept zu verstehen. Es wurden keine Optimierungen vorgenommen und lediglich das Grundlegende Verfahren umgesetzt. Unter anderem wurde das G-Buffer Layout und seine Größe nicht mit den Anforderungen skaliert. Bis auf dem Depth-Attachment mit 24-Bit sind alle anderen Attachments 32-Bit Float-Texturen. Wie in Abbildung~\ref{fig:gbuffer} zu sehen ist, werden 64-Bit der Kapazität des G-Buffers nicht genutzt. Auch kann die Größe der einzelnen Attachments weiter reduziert werden. Beispielsweise ist zur Speicherung der Komponenten eines Normalen-Vektors keine 32-Bit Float-Textur nötig. Auch wird im Attachment das den Spekular-Koeffizienten enthält nur ein Kanal genutzt. Dieser könnte auch im Alpha-Kanal des Attachement welches die Albedo-Color enthält gespeichert werden.

Auch der Rendering-Prozess kann deutlich verbessert werden. Jede Punktlichtquelle beleuchtet nur einen Bereich innerhalb ihres Radius. Trotzdem wird für jedes Fragment jede Lichtquelle in die Berechnung einbezogen, auch wenn diese keinen Beitrag zur Beleuchtung des Fragments gibt. Dies lässt sich Beispielsweise dadurch verhinden, das für jedes Licht ein Quadrat mit Kantenlänge seines Durchmessers im Screen-Space gerendert wird und nur auf diesem Quadrat mit dieser Lichtquelle die Beleuchtung berechnet. Die einzelnen Quadrate werden dann additiv geblendet um das korrekte Bild zu erhalten.

Außerdem werden bisher auch Lichtquellen in die Berechnung einbezogen die außerhalb des View-Frustums der Kamera liegen. Durch eine Art Frustum-Culling lässt sich auch hier Rechenleistung einsparen.
\chapter{Bedienung}
Kamera
\begin{itemize}
\item W - Vorwärtsbewegung
\item S - Rückwärtsbewegung
\item A - Seitwertsbewegung (links)
\item D - Seitwertsbewegung (rechts)
\end{itemize}

G-Buffer Attachments
\begin{itemize}
\item E - Ein-\textbackslash Ausblenden
\end{itemize}
%\nocite{*}
%--------------------------------------------------------------------------
\backmatter                        		% Anhang
%-------------------------------------------------------------------------
\bibliographystyle{geralpha}			% Literaturverzeichnis
%\bibliography{literatur}     			% BibTeX-File literatur.bib
%--------------------------------------------------------------------------
\printindex 							% Index (optional)
%--------------------------------------------------------------------------
\begin{appendix}						% Anhänge sind i.d.R. optional
\end{appendix}

\end{document}
